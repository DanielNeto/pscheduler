%
% pScheduler Developer's Guide
%

% TODO: Add titlepage to the options.
\documentclass[10pt,titlepage]{article}

\input pscheduler-tex.tex

\DRAFT

\title{pScheduler Plug-In Developer's Guide}
\author{The perfSONAR Development Team}


\begin{document}
\maketitle
\tableofcontents


\part{Introduction}

\section{Developing for pScheduler}

This guide provides the information required to build software modules
that add to pScheduler's capabilities.

\todo{Expand this.}


\section{Configuring a System for Development and Test}

Development and test is currently supported on Red Hat Enterprise
Linux or CentOS 6.{\it x}.

\todo{Is the model going to be that we develop on CentOS and do ports to Debian?}


\subsection{System Requirements}

The minimum recommended configuration for build and development is as
follows:

\begin{itemize}
\item 32- or 64-bit Intel CPU.  (64-bit recommended)
\item 2 GiB DRAM
\item 10 GB disk
\item Network connection which can reach the Internet, specifically
  the PostgreSQL YUM repository at {\tt yum.postgresql.org} and a RHEL
  or CentOS mirror.  (The perfSONAR development team will eventually
  make this repository available to sites unable to access non-R\&E
  networks.)
% TODO: Make PgSQL available on software.internet2.edu.
\end{itemize}


\subsection{Installation and Configuration}

Follow these steps to configure the build/development system:

\begin{enumerate}
\item Boot the Red Hat Enterprise Linux or CentOS 6.{\it x} image.
\item Configure the language, keyboard and storage devices
  appropriately.
\item Configure the network interface(s) to connect automatically.
\item Configure the time, root password, and disk usage appropriately.
\item Use the ``Basic Server'' configuration for installation.
\item Finish the installation and reboot.
\item Log in as \root.
\item \rootcommand{iptables -A INPUT -p tcp --dport 22 -j ACCEPT}
\item \rootcommand{iptables -A OUTPUT -p udp --sport 22 -j ACCEPT}
\item \rootcommand{service iptables save}
\item Copy the file {\tt scripts/system-prep} from the pScheduler
  source distribution onto the new machine and place it on the new
  system in {\tt /tmp}.
\item If the system is a virtual machine hosted on VirtualBox on Linux
  or OS X and you wish to have your account and home directory
  available, configure the guest with a shared folder for your home
  directory and named to match your login (e.g., {\tt /home/bob} as
  {\tt bob}).  Then edit {\tt system-prep}, uncomment and set the four
  environment variables at the top and attach the {\it VirtualBox
    Guest Additions} CD image.
\item \rootcommand{/tmp/system-prep}
\end{enumerate}

If the system is a virtual machine, this is a good point to create a
snapshot or export it as an appliance.


\subsection{Building and Installing pScheduler}

pScheduler is divided into multiple packages, some of which have
dependencies on others and each must be installed after being built.
You should, therefore, consider the build/development system
expendable and not do development on a production system.

Other than the installation of finished packages, the build process
does not produce by-products outside of the source directories.

Once you have a system configured as described above, build and
install pScheduler by doing the following:

\begin{enumerate}
\item Copy the pScheduler source distribution onto the system.
\item Become \root.
\item Unpack the source distribution.
\item \rootcommand{cd {\it pScheduler source directory}}
\item \rootcommand{make}
\end{enumerate}

Everything needed for pScheduler will be started on the next system
boot.

The effects of the installation can be un-done with \command{make
  uninstall}.


% -----------------------------------------------------------------------------
% -----------------------------------------------------------------------------


\part{The pScheduler Plug-In Architecture}

As its name hints, pScheduler's job is to do five things, all of which
center around making measurements:

\begin{enumerate}
\item Accept tasking from the outside.
\item Maintain a schedule of one or more times when the tasking should
  be carried out.  These events are called {\it runs}.
\item Carry out the tasking by executing the scheduled runs.
\item Receive run results.
\item Distribute the results to archives for processing and long-term
  storage.
\end{enumerate}

One of the key tenets of pScheduler's design is that it should have no
understanding of the measurements it schedules or the programs which
to carry them out.  This is accomplished through abstraction and a
system of plug-ins which can be installed in any combination on a
perfSONAR system.  By using plug-ins, perfSONAR's capabilities can be
expanded by its user community without having to rely on the core
development team's limited resrouces.  Software to handle measurements
not of broad enough interest to be part of the mainstream perfSONAR
distribution can be developed and maintained by those who need them.

In order to make this happen, pScheduler establishes three classes of
things with which it can interface:

\begin{itemize}
\item {\it Tests}, which are abstract descriptions of a measurement to
  be made and the results.
\item {\it Tools}, which are programs which, in single or multiple
  instances working together, make a measurement specified by a test.
\item {\it Archivers}, which send test results produced by tools
  elsewhere for long-term storage.
\end{itemize}

Each of these will be described in more detail in later sections.

\section{Concept of Operation}

\todo{Introductory paragraph}

This is the gist of what pScheduler does (or will do when completed):

\begin{enumerate}
\item On startup, pScheduler inventories the installed tests, tools
  and archivers.  For tools, this includes getting information on
  which tests they run.
\item A task, consisting of a test specification, scheduling,
  archiving and other administrative information is written and
  provided to pScheduler.
\item The test specfication is validated by the test whose type is
  specified in the task.
\item pScheduler consults all of the installed tools which have
  indicated they can run the test type in the task.  This information
  is passed back to the requester.
\item The requester, having done the same with the other
  participant(s) in the measurement, selects which tool will be used
  and submitted to pScheduler for scheduling.
\item At the scheduled time, the measurement is performed, producing a
  result.
\item The result is stored locally.
\item The results are retrieved from all participants and merged into
  a single result, which is stored for archiving.
\item Each result is sent to each of the archivers specified in the
  task, with repeating attempts after a failure.
\end{enumerate}


\section{Plug-In Concepts}

\subsection{Classes}

At the top level of the API are {\it classes}, each of which defines
an entity pScheduler requires to function.

There are three defined classes, which are the three items described
earlier (tests, tools and archivers).

\subsection{Methods}

Each class defines one or more {\it methods}, which carry out
functions specific to that class.

Methods are invoked across a process boundary (i.e., by running
programs external to pScheduler).  This almost completely decouples
pScheduler's technology stack from those of the components, providing
several benefits:

\begin{itemize}
\item Developers are free to select the right tool for the job 

developers to select the right tool for the job and allowing
significant evolution of one part without requiring the same of the
others.
\end{itemize}

Each method program will:

\begin{itemize}
\item Be installed in a defined location.
\item Have a defined name based on its function.
\item Accept input in a defined format.  For most programs, this will
  be JSON data.
\item Provide output on the standard output stream in a defined
  format.  For most programs, this will be JSON data.
\item Provide error output on the standard error stream as plain,
  human-readable text.
\item Exit with one of several defined status values, usually
  following the {\tt true}/{\tt false} convention or using values in
  the \headerfile{sysexits.h} header found on many systems.
\end{itemize}


\subsection{Implementations}
\todo{Write this.  (What was I going to put in this section?)}


\subsection{Directory Structure}

\note{This information is provided as background and should not be
  used in determining where files should be installed.  Specific
  instructions for use with the standard packaging systems for various
  OS distributions can be found later in this document.}

Generally, pScheduler's installation of files follows the conventions
put forth in the {\it Filesystem Hierarchy Standard}
(\url{http://www.pathname.com/fhs}).

A typical directory tree on a system with several tests, tools and
archivers installed might look like this:

\dirtree{%
.1 /usr.
.2 libexec.
.3 pscheduler.
.4 classes $\odot$.
.5 test $\odot$ {\it Implementations of tests}.
.6 idle.
.6 rtt.
.5 tool $\odot$ {\it Implementations of tools}.
.6 sleep.
.6 snooze.
.6 ping.
.6 owping.
.5 archiver $\odot$ {\it Implementations of archivers}.
.6 bitbucket.
.6 esmond.
.2 share.
.3 doc.
.4 pscheduler.
.5 test $\odot$ {\it Test documentation}.
.5 tool $\odot$ {\it Tool documentation}.
.5 archiver $\odot$ {\it Archiver documentation}.
}

The full paths of directories marked with the symbol $\odot$ should be
provided as macros for packaging system specifications to use in
ensuring correct placement of installed files.

\todo{Should we specify the name of a directory inside each
  test/tool/archiver where private code can live?  (Thinking about
  common JSON validator modules which would to be used for tool {\tt
    can-run} and {\tt run} methods.)  {\tt PRIVATE} might be a good
  candidate.}


% -----------------------------------------------------------------------------
% -----------------------------------------------------------------------------


\section{Plug-In Classes}

\todo{Need to add a ``Command Line Arguments'' item to the description
  of each method.}

\todo{Change the ``Exit Status'' sections to be enumerated so they're easier to read.}

%
% TEST
%
\subsection{Test}

\todo{Introductory material}

\todo{Design of Tests}

\todo{Work this in somewhere: Tests are not necessarily restricted to
  the subset of parameters the tools which implement it have in
  common, rather they should be the superset of all parameters.  This
  makes it possible to use tool-unique features without requiring that
  every tool understand every parameter.  Tools will seeing tests
  containing parameters they do not understand or are outside of their
  abilities will simply decline to perform them.}

\todo{Test Specification}

The specification for a round-trip time test might look like this:
\begin{lstlisting}[language=json,firstnumber=1]
{
  "schema": 1,
  "source": null,
  "target": "somehost.example.org",
  "interval": "PT3S",
  "size": 1024,
  "ttl": 100,
  "tos": "0x5a"
  }
}
\end{lstlisting}



Every test must implement the methods listed in the following sections.

The directory tree of a test implementation would look like this ({\tt
  *} denotes an executable file):
\dirtree{%
.1 /usr/libexec/pscheduler/classes/test.
.2 {\itt test-name}.
.3 enumerate*.
.3 participants*.
.3 spec-is-valid*.
}



\subsubsection{enumerate}

This method produces a description of the test in a standard format
used when pScheduler needs to determine what tests are installed.

\calloutitem{Standard Input} Ignored.

\calloutitem{Standard Output} JSON containing the following items:
\begin{itemize}
\item{\tt name} - A string serving as a short name for the test.  This
  value should be lowercase and match the name of the directory where
  the test class is installed.  Test names should be industry-standard
  terms rather than the names of the tools which carry them out.  For
  example, a test that determines how long it takes for packets to get
  from one host to another should be called {\tt latency} and not {\tt
    owamp}.  A test the determines the same for a round trip should be
  called {\tt rtt} (short for {\it round-trip time}) and not {\tt
    ping}.
\item{\tt description} - A short, textual description of what the test
  measures.
\item{\tt version} - A \jsonobject{Version}.
\item{\tt maintainer} - A \jsonobject{Maintainer}.
\end{itemize}

\example
\begin{lstlisting}[language=json,firstnumber=1]
{
    "name": "rtt",
    "description": "Round-trip time between hosts",
    "version": "1.0",
    "maintainer": {
        "name": "Example Development Team",
        "email": "plug-ins@example.org",
        "href": "http://www.example.org/plug-ins"
    }
}
\end{lstlisting}

\calloutitem{Standard Error} Empty if successful, plain text error
messages on error.

\calloutitem{Exit Status} {\tt 0} if successful, nonzero on error.

% TODO: At some point in the future, there will probably be an
% enumeration of what the test itself should look like for use by
% GUIs.


\subsubsection{participants}
This method produces a list of the systems participating in the test
based on the test specification.

\calloutitem{Standard Input} A JSON specification of the test.

\calloutitem{Standard Output} A JSON array of the participants, with
any {\tt null} values indicating the local host.  The order of this
list will match the numbering of participant roles specified in the
definition of the test.
\example
\begin{lstlisting}[language=json,firstnumber=1]
[ "sendinghost.example.com", "receivinghost.example.com" ]
\end{lstlisting}

\calloutitem{Standard Error} Empty if successful, plain text error
messages on error.

\calloutitem{Exit Status} {\tt 0} if successful, nonzero on error.



\subsubsection{spec-is-valid}
This method determines whether not a test specification is valid.

Validation of a parameter should be according to whether or not its
value is usable in a test, not whether or not any of the tools
actually implement it.  For example, a test with a {\tt port}
parameter representing a UDP or TCP port should accept numeric values
in the range {\tt 0} to {\tt 65535} and perhaps strings which can be
translated to port numbers using {\tt getservbyname(3)}.  Tools will
accept or decline individual tests based on their own restrictions.

\calloutitem{Standard Input} A JSON test specification.

\calloutitem{Standard Output} None.

\calloutitem{Standard Error} Empty if successful, plain text error
messages on error.

\calloutitem{Exit Status} {\tt 0} if the test specification was valid,
            {\tt 1} if not, {\tt 2} if there was some other problem
            unrelated to the input.



\subsubsection{result-is-valid}
This method determines whether or not a result produced by a tool for
this test conforms to the expected format.

\calloutitem{Standard Input} JSON results for the test as produced by
a tool.

\calloutitem{Standard Output} None.

\calloutitem{Standard Error} Empty if successful, plain text error
messages on error.

\calloutitem{Exit Status} {\tt 0} if the result was valid, {\tt 1} if
not, {\tt 2} if there was some other problem unrelated to the input.



\subsubsection{cli-to-spec}
\todo{Validate command-line options and output a JSON spec for the
  test.  This will be needed once we start scheduling tests from the
  command line.  (There should also be the option to take the spec
  directly as JSON from standard input.)}

\todo{The existing test implementations don't have this yet.}


\subsubsection{spec-to-cli}
\todo{Produce command-line options from a JSON spec in a form that can
  be pasted into a shell script.  Not strictly necessary, but will be
  a nice convenience for users.}

\todo{The existing test implementations don't have this yet.}


%
% FEATURE
%
\subsection{Feature}

\todo{This feature will not be present in early releases.}

\todo{Write this.  Features will be what take care of altering the
  environment using chains of exec'd programs prior to running a tool
  and tearing down afterward.  This is to solve the MDVPN problem.}

\todo{Consider a different name for these.  Modifiers?  Preparers?}



%
% TOOL
%
\subsection{Tool}
\todo{Write this.}

\subsubsection{enumerate}

This method produces a description of the tool in a standard format
used when pScheduler needs to determine what tools are installed.

\calloutitem{Standard Input} Ignored.

\calloutitem{Standard Output} JSON containing the following items:
\begin{itemize}
\item{\tt name} - A string serving as a short name for the tool.  This
  value should be lowercase and match the name of the directory where
  the tool class is installed.  The name should reflect the software
  used, such as {\tt traceroute}, {\tt tracepath} or {\tt mtr}.
\item{\tt description} - A short, textual description of what the tool
  does.
\item{\tt version} - A \jsonobject{Version}.
\item{\tt tests} - An array of strings listing the tests which the
  tool is capable of accommodating.  Including a class here does not
  imply that the tool is required to actually perform the test, only
  that it should be asked if it can perform tests of those listed.
\item{\tt preference} - An integer used to determine the order in
  which tools are selected for tests, with larger numbers indicating
  more oreference.  This allows current, better-supported tools to be
  considered first while still allowing those that are deprecated to
  be used if necessary.  Nominally, the value for most tools should be
  {\tt 0} and deprecated tools should use a negative number.  This
  value is considered advisory only, as some measurements may specify
  that a specific tool be used to conduct a measurement.
\item{\tt maintainer} - A \jsonobject{Maintainer}.
\end{itemize}

\example
\begin{lstlisting}[language=json,firstnumber=1]
{
    "name": "mtr",
    "description": "Determine network path using MTR",
    "version": "1.0",
    "tests": [ "path" ],
    "preference": 0,
    "maintainer": {
        "name": "Example Development Team",
        "email": "plug-ins@example.org",
        "href": "http://www.example.org/plug-ins"
    }
}
\end{lstlisting}

\calloutitem{Standard Error} Empty if successful, plain text error
messages on error.

\calloutitem{Exit Status} {\tt 0} if successful, nonzero on error.



\subsubsection{can-run}
This method determines whether or not the tool can run a specified test.  The test must be 

\calloutitem{Standard Input} A JSON test specification.

\calloutitem{Standard Output} None.

\calloutitem{Standard Error} Empty if successful, plain text error
messages on error.

\calloutitem{Exit Status} {\tt 0} if the tool is willing to run the
specified test, {\tt 1} if not, {\tt 2} if there was an error related
to the input and {\tt 3} for an error unrelated to the input.


\subsubsection{run}
This method carries out the test as one of its participants.
\todo{Could use some examples.}

\calloutitem{Standard Input} A JSON run specification containing the following items:
\todo{Contents TBD.  Will include a participant number, schedule and the test spec.}

\calloutitem{Standard Output} None.

\calloutitem{Standard Error} Empty if successful, plain text error
messages on error.

\calloutitem{Exit Status} {\tt 0} if the test ran successfully, {\tt
  1} if there was a test-related error (i.e., the test could run but
was unable to produce results because a remote participant failed to
answer) or {\tt 2} if there was an error unrelated to the test.



%
% ARCHIVER
%
\subsection{Archiver}
\todo{Write this.}

\subsubsection{enumerate}

This method produces a description of the archiver in a standard
format used when pScheduler needs to determine what archivers are
installed.

\calloutitem{Standard Input} Ignored.

\calloutitem{Standard Output} JSON containing the following items:
\begin{itemize}
\item{\tt name} - A string serving as a short name for the archiver.
  This value should be lowercase and match the name of the directory
  where the archiver implementation is installed.  Archiver names
  should reflect the destination, such as {\tt bitbucket} for a data
  sink or {\tt esmond} for an archiving system like {\it esmond}.
\item{\tt description} - A short, textual description of what the
  archiver does.
\item{\tt version} - A \jsonobject{Version}.
\item{\tt maintainer} - A \jsonobject{Maintainer}.
\end{itemize}

\example
\begin{lstlisting}[language=json,firstnumber=1]
{
    "name": "bitbucket",
    "description": "Dump results into the bit bucket.",
    "version": "1.0",

    "maintainer": {
        "name": "perfSONAR Development Team",
        "email": "info@perfsonar.net",
        "href": "http://www.perfsonar.net"
    }
}
\end{lstlisting}

\calloutitem{Standard Error} Empty if successful, plain text error
messages on error.

\calloutitem{Exit Status} {\tt 0} if successful, nonzero on error.


\todo{Probably need a method to check whether an archive spec is valid
  for this archiver.}

\subsubsection{archive}

This method sends the results of a run to the archiver.

\calloutitem{Standard Input} JSON containing the following items:
\begin{itemize}
\item \todo{TBD}
\end{itemize}
\example
\begin{lstlisting}[language=json,firstnumber=1]
{
    "TODO": "TBD"
}
\end{lstlisting}

\calloutitem{Standard Output} None.

\calloutitem{Standard Error} Empty if successful, plain text error
messages on error.

\calloutitem{Exit Status} {\tt 0} if successful, nonzero on error.




% -----------------------------------------------------------------------------

\part{Building Plug-Ins}

\section{Implementation}
\todo{Write this.}

\section{Packaging}

\subsection{Red Hat Enterprise Linux, CentOS and Fedora}
\todo{Write this.}

The RPM specs for the {\tt pscheduler-archiver-bitbucket}, {\tt
  pscheduler-test-idle} and {\tt pscheduler-tool-sleep} packages are
recommended to be used as templates for new development.

With {\tt pscheduler-rpm} installed on the system, the following
macros are available for use in RPM specifications:
\begin{center}
  \begin{tabular}{ll} 
    % Tests
    {\tt _pscheduler_test_libexec} & Test implementations \\
    {\tt _pscheduler_test_doc} & Test documentation \\
    % Tools
    {\tt _pscheduler_tool_libexec} & Tool implementations \\
    {\tt _pscheduler_tool_doc} & Tool documentation \\
    % Archivers
    {\tt _pscheduler_archiver_libexec} & Archiver implementations \\
    {\tt _pscheduler_archiver_doc} & Archiver documentation \\
  \end{tabular}
\end{center}

Test, tool and archiver implementations should be installed in a
subdirectory matching their names.  For example,
{\tt\%\{_pscheduler_tool_libexec\}/owamp} would be the recommended
way to specify the installation directory for the {\tt owamp} tool.

\subsection{Debian}
\todo{Write this after Antoine has had a chance to package a few
  things.}

\subsection{OS X}
\todo{Do we want to support OS X?  Nothing in pScheduler should be
  Linux-specific.  Some tools may need to understand what environment
  they're running in when invoking test programs because command line
  switches or other things may be different (e.g., With {\tt ping(8)},
  the switch to change TOS is {\tt -Q} where OS X uses {\tt -z}).
  This will either need to be accommodated by making the tools smart
  enough to detect and handle the differences or having multiple
  versions.  Either way, there may be a need for a list of supported
  OS names to be in the enumeration.  (Use the output of {\tt uname
    -s} for this.)}

\end{document}
