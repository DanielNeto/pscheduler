%
% pScheduler Test Guide for Simplestream
%

\def\testname{simplestream}


% TODO: Add titlepage to the options.
\documentclass[10pt]{article}

\input pscheduler-tex.tex

\DRAFT

\title{pScheduler Test Guide: {\it \testname}}
\author{The perfSONAR Development Team}


\begin{document}
\maketitle


%
% INTRODUCTION
%

\section{Introduction}

The {\tt \testname} test establishes a TCP connection between two
participants (a {\it sender} and a {\it receiver}), sends some test
data and then closes the connection.  This test is intended for use in
development and testing and not for use in producion.

This document describes schema version {\tt 1}.

\subsection{Participants}

This test includes the following participants:

\begin{center}
\begin{tabular}{|c|c|}
\hline
{\bf Number} & {\bf Role} \\
\hline
{\tt 0} & Sender \\
{\tt 1} & Receiver \\
\hline
\end{tabular}
\end{center}



%
% TEST SPECIFICATION
%

\section{Test Specification Format}

\subsection{Description}

The test specification consists of a single JSON object containing the
pairs below.  \seejson

\typeditem{schema}{Cardinal} The schema version of this specification.

\typeditem{receiver}{Host} The network address or hostname of the receiver.

\typeditem{dawdle}{Duration} Optional maximum amount of time the sender
may intentionally wait between the start of the test and establishing
the connection to the receiver.  This value must be less than {\tt
timeout}.  If not provided, the sender should begin the test
immediately.  Tools implementing this test may select a random
duration between {\tt P0} and this value.

\typeditem{test-material}{String} Optional text to be sent across the
connection.  If not provided, the sender may send any data it chooses.

\typeditem{timeout}{Duration} The amount of time the receiver will wait
before declaring the test a failure.  The default will be determined
by the tool executing the test.


\subsection{Example}
\begin{lstlisting}[language=json]
{
    "schema": 1
    "receiver": "test19.example.org",
    "dawdle": "PT2S",
    "test-material": "The quick brown fox jumped over the lazy dog's back.",
    "timeout": "PT5S"
}
\end{lstlisting}



%
% RESULT FORMAT
%

\section{Result Format}

\subsection{Description}
The result consists of a single JSON object containing the pairs
below.  \seejson

\typeditem{schema}{Cardinal} The schema version of the result.

\typeditem{dawdled}{Duration} The time the sender spend dawdling
before opening the connection.

\typeditem{elapsed-time}{Duration} The time elapsed between the start
of the test and when the sender closed the connection, as measured by
the receiver.

\typeditem{succeeded}{Boolean} Whether or not the test was successful.

\typeditem{dawdled}{Duration} How long the sender dawdled before
connecting to the receiver.

\typeditem{sent}{String} The test material sent by the sender.

\typeditem{received}{String} The test material received by the receiver.

\typeditem{elapsed-time}{Duration} The amount of time that elapsed between
the time the reciever started listening for a connection and the


\subsection{Example}
\begin{lstlisting}[language=json]
{
    "schema": 1,

    "succeeded": true,
    "dawdled": "PT0.23498S",
    "sent": "This is a test.",
    "received": "This is a test.",
    "elapsed-time": "PT3.14S"
}
\end{lstlisting}


\end{document}
