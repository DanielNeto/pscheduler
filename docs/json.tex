%
% pScheduler JSON Style Guide and Dictionary
%

% TODO: Add titlepage to the options.
\documentclass[10pt]{article}

\input pscheduler-tex.tex

\DRAFT

\title{pScheduler JSON Style Guide and Type Dictionary}
\author{The perfSONAR Development Team}


\begin{document}
\maketitle

%
% Utilities
%

\def\source#1#2{{\bf #1:}  \url{#2}}


%
% INTRODUCTION
%

\section{Introduction}
\todo{Write this, mention ECMA-404.}

Sources of reference for the standards referred to multiple times by
this document can be found in section \autoref{standards}, {\it
  \nameref{standards}}.


%
% JSON STYLE
%

\section{JSON Style}
\todo{Write this.}

These guidelines apply to the definition of JSON for
pScheduler-specific applications.  For JSON values which follow a
different, established standard, that standard will be used instead.

\subsection{Pair Names}
Pair names will be lowercase with words separated by hyphens.

For reasons described in \autoref{comments}, {\it
  \nameref{comments}}.  No name in a pair will begin with the number
sign (``{\tt \#}'')

\example\ {\tt this}, {\tt that} and {\tt the-other}.


\subsection{Comments}\label{comments}

In the interest of supporting comments without violating the standard,
pScheduler's internal JSON parser will discard any pair it encounters
whose name begins with the number sign (``{\tt \#}'').

\example
\begin{lstlisting}[language=json]
{
    "uncommented": "This pair will remain",
    "#commented": "This pair will be removed",
    "array": [
        "This will remain",
        "#This will, too, because it is not a pair name"
    ]
}
\end{lstlisting}

Developers wishing to use this loader in plugins written in Python
should refer to the {\it pScheduler Plug-In Developer's Guide} for
documentation on the {\tt json_load()} function.  \todo{This
  documentation does not exist yet.}



\subsection{Ranges}\label{ranges}
A number of types have range subtypes which consist of an object with
an {\tt upper} and {\tt lower} pair.







%
% TYPE DICTIONARY
%

\section{Type Dictionary}

This section covers simple and compound types used in pScheduler.

For descriptions of standard types used by the limit system, refer to
{\it The pScheduler Limit System}.


\subsection{AnyJSON}
Any valid JSON value as defined by ECMA 404.



\subsection{ArchiveSpecification}
Specification of an archive destination and supporting data, including these items:

\typeditem{archiver}{String} The name of the archiver to be used.  The
name provided must be one which is installed on the system.

\typeditem{data}{AnyJSON} Data specific to the type of archiver which
direct its behavior.  If not provided, {\tt null} will be used as the
default.  Note that the REST API will change the value of any element
whose name begins with an underscore to {\tt null} before returning it
as part of a query.  This mechanism allows for information that needs
to be kept private (e.g., authentication keys) to be provided as
archiver data.

\example
\begin{lstlisting}[language=json]
{
    "archiver": "syslog",
    "data": {
        "facility": "local7",
	"level": "info",
        "tag": "nothing-important"
    }
}
\end{lstlisting}


\typeditem{ttl}{Duration} An optional amount of time after which the
result should be discarded if no attempts have been made to archive
it.  If not specified, the default behavior will be to keep it in the
queue as long as the result is still in the system.




\subsection{Array}
An valid JSON array as defined by ECMA 404.


\subsection{AS}
An object containing information about an autonomous system (AS).  It
contains the these elements:

\typeditem{number}{Cardinal} The AS's number.

\typeditem{owner}{String} Optional free-form information about the
organization that owns the AS.

\example
\begin{lstlisting}[language=json]
{
    "number": 65432,
    "owner": "Fictitous Networks, Inc."
}
\end{lstlisting}




\subsection{Boolean}
A JSON boolean value as defined by ECMA 404.

\example\ \true\ and \false.  (Note that these are also
the only two valid values.)


\subsection{Cardinal}
A subset of {\tt number} as defined by ECMA 404 with the restriction
that the value be positive and contain no decimal point.

\example\ {\tt 1}, {\tt 28} and {\tt 906}.



\subsection{CardinalList}
An array of zero or more {\it Cardinal} elements.

\example\ {\tt [ 72, 897 ]} and {\tt [1, 2, 3]}


\subsection{CardinalRange}
A range of {\tt Cardinal} values.  See note about ranges in
introductory material.




\subsection{CardinalZero}
A subset of {\tt number} as defined by ECMA 404 with the restriction
that the value not be negative and contain no decimal point.

\example\ {\tt 0}, {\tt 37} and {\tt 1732}.


\subsection{CardinalZeroList}
An array of zero or more {\it CardinalZero} elements.

\example\ {\tt [ 72, 897 ]} and {\tt [1, 2, 3]}


\subsection{CardinalZeroRange}
A range of {\tt CardinalZero} values.  See note about ranges in
introductory material.



\subsection{ClockState}
A JSON object containing information about the state of a host's
clock.

\typeditem{time}{Timestamp} The current time as reported by the clock.

\typeditem{synchronized}{Boolean} Whether or not the clock is
synchronized to another source.

\typeditem{source}{String} An arbitrary string indicating the source
of time synchronization.  The recommended values are {\tt cdma}, {\tt
  gps}, {\tt ntp} and {\tt ptp}

\typeditem{reference}{String} An arbitrary string describing the
reference being used to derive the time.

\typeditem{offset}{Duration} The amount of error the timekeeping
system believes may be present in the provided time.


\example
\begin{lstlisting}[language=json]
{
    "time": "2017-02-05T08:41:50.424743-08:00",
    "synchronized": true,
    "source": "ntp",
    "reference": "secondary reference (2) from 10.19.54.6",
    "offset": 6.6041946411132812e-05
}
\end{lstlisting}




\subsection{Duration}
A length of time as defined by a subset the ISO 8601 standard.  All
types of intervals in the standard are supported except years (e.g.,
{\tt P2Y}) and months (e.g., {\tt P7M}), which are inexact.  Use of
either as part of a duration will result in an error.

\example\ {\tt P3D}, {\tt PT15M} and {\tt P3DT2H6M18S}.


\subsection{DurationRange}
A JSON object containing a range of durations.

\typeditem{lower}{Duration} The lower end of the range.

\typeditem{upper}{Duration} The upper end of the range.

The upper end of the range must be greater than the lower.


\example
\begin{lstlisting}[language=json]
{
    "lower": "PT30S",
    "upper": "PT1M"
}
\end{lstlisting}



\subsection{Email}
An electronic mail address as defined by RFC 5322, section 3.4.

\example\ {\tt alice@finance.example.com} and {\tt
  bob.smith@eng.example.org}.

\source{IETF}{https://tools.ietf.org/html/rfc5322}


\subsection{Geographic Position}
A string representing coordinates on the Earth in the format defined
by ISO 6709, Annex H.  The trailing slash mandated by the standard is
optional but should be accepted if provided.

\example\ {\tt +48.8577+002.295/} and {\tt +40.6894-074.0447/}.

\source{ISO}{http://www.iso.org/iso/home/store/catalogue_tc/catalogue_detail.htm?csnumber=39242}\\
\source{Wikipedia}{https://en.wikipedia.org/wiki/ISO_6709}





\subsection{Host}
The address of a host on the Internet as a string.

{\bf IPv4.}  IPv4 addresses should be specified in the dotted-quad
notation.  (Note that this format was never standardized, although it
did first appear in RFC 790.)  Alternately, a hostname,
partially-qualified domain name or fully-qualified domain name (FQDN)
may be used.

{\bf IPv6.}  IPv6 addresses should be specified in the format
described by RFC 5952.  Alternately, a hostname, partially-qualified
domain name or fully-qualified domain name (FQDN) may be used,
prefixed with {\tt ipv6:}.  \todo{See if this is a good thing to do.}

\example\ {\tt test.example.org}, {\tt 10.48.209.7}, {\tt
  [2001:db8:a0b:12f0::1]} and {\tt ipv6:www6.example.net}.

\source{IETF}{https://tools.ietf.org/html/rfc790}\\
\source{IETF}{https://tools.ietf.org/html/rfc5952}


\subsection{ICMPError}
\todo{This is a standard value; should not be a type.}
A string representing one of the possible ICMP type 3 errors defined by RFC 792 or RFC 1812:

\begin{itemize}
\item {\tt net-unreachable}
\item {\tt host-unreachable}
\item {\tt protocol-unreachable}
\item {\tt port-unreachable}
\item {\tt fragmentation-needed-and-df-set}
\item {\tt source-route-failed}
\item {\tt destination-network-unknown}
\item {\tt destination-host-unknown}
\item {\tt source-host-isolated}
\item {\tt destination-network-administratively-prohibited}
\item {\tt destination-host-administratively-prohibited}
\item {\tt network-unreachable-for-type-of-service}
\item {\tt icmp-destination-host-unreachable-tos}
\item {\tt communication-administratively-prohibited}
\item {\tt host-precedence-violation}
\item {\tt precedence-cutoff-in-effect}
\end{itemize}

\source{IETF}{https://tools.ietf.org/html/rfc792}\\
\source{IETF}{https://tools.ietf.org/html/rfc1812}



\subsection{Integer}
A subset of {\tt number} as defined by ECMA 404 with the restriction
that the value contain no decimal point or scientific notation.

\example\ {\tt 0}, {\tt 3713} and {\tt -6264}.






\subsection{IPAddress}
A valid IPv4 or IPv6 address.  See {\tt IPv4} and {\tt IPv6}.


\subsection{IPCIDR}
A valid IPv4 or IPv6 CIDR block.  See {\tt IPv4CIDR} and {\tt IPv6CIDR}.


\subsection{IPTOS}
An IP type-of-service octet per RFC 2474, section 3, represented as a
decimal number.


\example\ {\tt 63} and {\tt 205}.


\subsection{IPv4}
A valid IPv4 address.

\example\ {\tt 192.168.1.5} and {\tt 10.38.19.7}.


\subsection{IPv4CIDR}
A valid IPv4 CIDR block, consisting of an IPv4 network address, a
slash and a {\tt Cardinal} indicating the number of bits in the
netmask.

\example\ {\tt 192.168.1.0/24} and {\tt 10.0.0.0/8}.


\subsection{IPv6}
A valid IPv6 address.

\example\ {\tt  2001:0db8:0a0b:12f0:0000:0000:0000:0001} and {\tt 2001:db8:1234:67e4::1}.


\subsection{IPv6CIDR}
A valid IPv6 CIDR block, consisting of an IPv6 network address, a
slash and a {\tt Cardinal} indicating the number of bits in the
netmask.

\example\ {\tt 2001:0db8::/51} and {\tt 2001:0db8:dead:beef/64}.



\subsection{IPPort}
A valid port number for use with the TCP/IP and UDP/IP protocols.
Valid values are of type {\tt Integer} in the range {\tt 0} to {\tt
  65535}.

\example\ {\tt 23}, {\tt 80} and {\tt 23842}.



\subsection{JQTransformSpecification}
An object containing the specification of a {\tt jq} filter for JSON
input.

\typeditem{script}{String} The script to be used in transforming the
input.  The syntax of the script is described at the web site listed
below.

\typeditem{output-raw}{Boolean} {\tt true} if the items produced by
the script should be converted into strings and joined with newlines.

\example
\begin{lstlisting}[language=json]
{
    "script": "\"The Foo value is \\(.foo)\""
}
\end{lstlisting}

Note that because the script is being encoded as a JSON string, all
characters special to JSON must be properly escaped.  These characters
are quotes and the backslash.

\source{JQ}{https://stedolan.github.io/jq}




\subsection{Maintainer}

An object containing identifying and contact information about the
maintainer of a piece of pSchedler software.  It contains these
elements:

\typeditem{name}{String} The name of the maintainer.

\typeditem{email}{Email} The maintainer's email address.

\typeditem{href}{URL} A URL where more information on the sofware or
the maintainer's organization can be found.

\example
\begin{lstlisting}[language=json]
{
    "name": "Example Software Development Team",
    "email": "software@example.org",
    "href": "http://www.example.org/software"
}
\end{lstlisting}



\subsection{NameVersion}

An object containing a name and version of some system item.

\typeditem{name}{String} The name of the item.

\typeditem{version}{Version} The version number of the item.

\example
\begin{lstlisting}[language=json]
{
    "name": "foomatic",
    "version": "7.2",
}
\end{lstlisting}



\subsection{Number}
A {\tt number} as defined by ECMA 404.

\example\ {\tt 47}, {\tt -19.3} and {\tt 6.022e23}.


\subsection{Numeric}
A numeric value, consisiting of a {\tt Number} or {\tt SINumber}.

\subsection{NumericRange}
A JSON object containing a range of {\tt Numeric} values.




\subsection{ParticipantResult}
Information about a participant in a test and the result it produced.

\typeditem{participant}{Host} The host which participated in the test
and produced the result.

\typeditem{result}{AnyJSON} The test-specifc partial result produced
by the participant.



\subsection{Probability}
A {\tt Number}, restricted to the range of {\tt [0.0, 1.0]}.

\example\ {\tt 0.2874}, {\tt 0} and {\tt 0.31415}.


\subsection{ProbabilityRange}
A JSON object containing a range of {\tt Probability} values.



\subsection{RunResult}
An object containing all information about a run of a test.

\typeditem{id}{UUID} A globally-unique identifier for the run that
produced this result.

\typeditem{schedule}{TimeInterval} When the run was scheduled to start
and how long it was scheduled to run.

\typeditem{test}{TestSpecification} The test specification as provided
when the task was instantiated.

\typeditem{tool}{NameVersion} The tool which conducted the
test.

\typeditem{participants}{Array} An array of {\tt ParticipantResult}
objects, one for each participant in the test.

\typeditem{result}{AnyJSON} The full, merged result of the test in a
test-specific format.  (See the pScheduler Test Guide for the test in
question for details.)

\example
\begin{lstlisting}[language=json]
{
    "id": "3c235f6e-672f-d219-f00f-54d0398914eb",
    "schedule": {
        "start": "2016-02-08T17:35:03-05",
        "duration": "PT30S"
    },
    "test": {
        "type": "foo",
        "spec": { ... Test Specification ... }
    },
    "tool": "foomatic",
    "participants": [
        {
            "host": "ps34.example.org",
            "result": { ... Participant 0 Result ... }
        },
        {
            "host": "ps2.example.org",
            "result": { ... Participant 1 Result ... }
        }
    ],
    "result": { ... Result ... },
}
\end{lstlisting}


\subsection{ScheduleSpecification}
An object containing a description of how runs of a test should be scheduled:

\typeditem{start}{TimestampAbsoluteRelative} When the first run of the
test should be scheduled.  If not provided, some reasonable default
close to the current time will be used.

\typeditem{slip}{Duration} The amount of time the start time of each
run may be delayed.

\typeditem{randslip}{Float} A fraction in the range $[0.0, 1.0]$
indicating how much to randomly vary the start time of each run within
any remaining slip after scheduling.

\typeditem{repeat}{Duration} How often runs should be repeated.  If
not provided, the test will be run exactly once.

\typeditem{until}{TimeStampAbsoluteRelative} When repeated runs of the
test should end.  If not provided or the special value {\tt forever}
is provided, repeats will continue forever or until other factors
cause them to stop.  This value may not be provided without a {\tt
  repeat}.

\typeditem{max-runs}{Cardinal} The maximum number of successful runs
of the test that may occur.  If not provided, the number of runs will
not be counted.  This value may not be provided without a {\tt
  repeat}.

\example
\begin{lstlisting}[language=json]
{
    "start": "@P1D",
    "slip": "PT30M",
    "randslip": 0.5,
    "repeat": "PT1H",
    "until": "2016-06-01T06:00:00",
    "max-runs": 100
}
\end{lstlisting}


\subsection{RetryPolicyEntry}
A segment of a retry policy containing these items:

\typeditem{attempts}{Cardinal} The number of times that an attempt
should be made, waiting for time {\tt wait} (see below) between each
attempt.

\typeditem{wait}{Duration} The amount of time to wait between attempts.

\example
\begin{lstlisting}[language=json]
{
    "attemts": 5,
    "wait": "PT1M"
}
\end{lstlisting}




\subsection{SINumber}
An integer expressed as an {\tt Integer} or a string with SI units.
The string will contain a {\tt number} as defined by ECMA 404 without
any exponentiation (e.g., {\tt 38.5} but not {\tt 3.85e7}), followed
optionally by suffix indicating magnitude.  The suffix may be any of
the units listed in this table:

\begin{center}
  \begin{tabular}{|c|c|c|c|c|c|}
    \hline
    \multicolumn{3}{|c}{{\bf ISO 80000-1 Decimal Units}} & \multicolumn{3}{|c|}{{\bf IEC 60027 Binary Units}} \\
    \hline
    {\bf Unit} & {\bf Prefix} & {\bf Multiplier} & {\bf Unit} & {\bf Prefix} & {\bf Multiplier} \\
    \hline
    {\tt k} & kilo & $1000^{1}$ & {\tt Ki} & kibi & $1024^{1}$ \\
    {\tt M} & mega & $1000^{2}$ & {\tt Mi} & mebi & $1024^{2}$ \\
    {\tt G} & giga & $1000^{3}$ & {\tt Gi} & gibi & $1024^{3}$ \\
    {\tt T} & tera & $1000^{4}$ & {\tt Ti} & tebi & $1024^{4}$ \\
    {\tt P} & peta & $1000^{5}$ & {\tt Pi} & pebi & $1024^{5}$ \\
    {\tt E} & exa & $1000^{6}$ & {\tt Ei} & exbi & $1024^{6}$ \\
    {\tt Z} & zetta & $1000^{7}$ & {\tt Zi} & zebi & $1024^{7}$ \\
    {\tt Y} & yotta & $1000^{8}$ & {\tt Yi} & yobi & $1024^{8}$ \\
    \hline
  \end{tabular}
\end{center}

Whitespace and case are ignored.

\example\ {\tt 1234},  {\tt "19.2k"}, {\tt "12G"}, {\tt "128Ki"} and {\tt "3.1415Pi"}

Note to developers: The pScheduler Python module contains a {\tt
  si_as_integer()} function which will convert an integer or string
for this type to an integer.

\source{ISO}{http://www.iso.org/iso/catalogue_detail?csnumber=30669}\\
\source{IEC}{https://webstore.iec.ch/publication/97}\\
\source{Wikipedia}{https://en.wikipedia.org/wiki/Binary_prefix}



\subsection{String}
A {\tt string} as defined by ECMA 404.

\example\ {\tt "Hello, World!"} and {\tt "He said
  \textbackslash"hello\textbackslash" as he went by."}


\subsection{StringMatch}
A description for performaing matches on a string consisting of a JSON
object containing these elements:

\typeditem{style}{string} - The type of match to be attempted.  Must
be one of {\tt exact} (strings must match exactly), {\tt contains}
(string must contain the match) or {\tt regex} (string must match a
regular expression).

\typeditem{match}{String} - String or regular expression to match.

\typeditem{case-insensitive}{Boolean} - Match in a case-insensitive
way.

\typeditem{invert}{Boolean} - Invert the result if {\tt true}

\example
\begin{lstlisting}[language=json]
{
    "style": "regex",
    "match": "^(apple|banana|cherry)\$",
    "case-sensitive": true,
    "invert": true
}
\end{lstlisting}



\subsection{TaskSpecification}
Specification of a task consisting of a JSON object containing these
elements:

\typeditem{\tt schema} - A {\tt Cardinal} indicating the version
of the structure.

\typeditem{test}{TestSpecification} The test to be carried out.

\typeditem{tool}{array} The tool which will be used to carry out that
task.  This is valid to provide but may be overwritten internally; see
{\tt tools}, below.

\typeditem{tools}{array} An array of strings naming the tools which
may be used to carry out a test in order if preference.

\typeditem{schedule}{ScheduleSpecification} An object describing how
runs of the test are to be scheduled.

\typeditem{archives}{Array} A list of archivers to which the result of
each run of the test should be sent.  Each element is an {\tt
  ArchiveSpecification}.

\typeditem{reference}{AnyJSON} An optional block of arbitrary JSON
provided by the tasker for reference purposes, such as adding tags to
a task for later query.


\example
\begin{lstlisting}[language=json]
{
    "schema": 1,
    "test": {
        "type": "idle",
	"spec": {
	    "schema": 1,
	    "duration": "PT1M"
            "parting-comment": "That's all, folks!"
	}
    },
    "tools": [ "snooze", "sleep" ],
    "schedule": {
        "start": "@PT1H",
        "slip": "PT15M",
        "repeat": "1H"
    },
    "archives": [
        {
            "name": "syslog",
            "data": {
                "facility": "local7",
        	"level": "info",
                "tag": "nothing-important"
        },
        {
            "name": "bitbucket",
        }
    ],
    "reference": {
        "tags": [ "this", "that" ],
        "pie": 3.14
    }
}



    ]
}
\end{lstlisting}




\subsection{TestSpecification}
A test specification consisting of a JSON object containing these
elements:

\typeditem{test}{String} The name of the test being carried out.

\typeditem{spec}{AnyJSON} The specification of the test parameters.
This must be valid for the test type specified by {\tt test}.  Note
that the REST API will change the value of any element whose name
begins with an underscore to {\tt null} before returning it as part of
a query.  This mechanism allows for information that needs to be kept
private (e.g., authentication keys) to be provided as a test
parameter.

\example
\begin{lstlisting}[language=json]
{
    "test": "idle",
    "spec": {
        "duration": "PT15S",
        "parting-comment": "That's all, folks!"
     }
}
\end{lstlisting}




\subsection{TimeInterval}
\todo{Is this being used anywhere?}
An interval of time consisting of a JSON object containing these
elements:

\typeditem{start}{Timestamp} The beginning of the time range.  This
value is inclusive.

\typeditem{duration}{Duration} The length of the time rage.

Note that when constructing a range of times by adding {\tt duration}
to {\tt start}, the calculated upper end of the range should be
exclusive (e.g., {\tt 2015-04-19T00:00:00 + PT1H} adds up to {\tt
  2015-04-19T01:00:00} but should cover all of the midnight hour and
not the {\tt 01:00} hour).

\example
\begin{lstlisting}[language=json]
{
    "start": "2016-02-16T13:44:18",
    "duration": "PT1M30S"
}
\end{lstlisting}


\subsection{TimeRange}
\todo{Is this being used anywhere?}
A range of times consisting of a JSON object containing these
elements:

\typeditem{lower}{Timestamp} The beginning of the time range.  This
value is inclusive.

\typeditem{upper}{Timestamp} The end of the time range.  This
value is exclusive.

\example
\begin{lstlisting}[language=json]
{
    "lower": "2016-01-19T15:00:00",
    "upper": "2016-01-19T15:30:00"
}
\end{lstlisting}


\subsection{Timestamp}
A date and time specified as defined the ISO 8601 standard.

\example\ {\tt 2015-02-27} and {\tt 2015-08-31T09:00:00-05}.



\subsection{TimestampAbsoluteRelative}
A date and time specified in one of the following formats:
\begin{itemize}
\item Absolutely, specified as a {\tt Timestamp}.
\item Relative to the current time, specified as a {\tt Duration}.
\item Time-aligned to the next even increment of time, specified as
  {\tt Duration} preceded by an at sign ({\tt @}) and calculated
  relative to the system's local time zone.  {\tt @PT1H} means the top
  of the next hour from now, {\tt @P1D} means midnight tomorrow.
\end{itemize}

\example\ {\tt 2016-05-04T02:50:15}, {\tt PT30M} and {\tt @P1D}



\subsection{URL}
\todo{Change this to URI}
A Uniform Resource Locator, Uniform Resource Identifier or Uniform
Resource Name specified as defined by RFC 3986.

\example\ {\tt http://www.example.org} and {\tt http://www.perfsonar.net}.

\source{IETF}{https://tools.ietf.org/html/rfc3986}


\subsection{UUID}
A universally-unique identifier as defined in RFC 4122.

\source{IETF}{https://www.ietf.org/rfc/rfc4122.txt}


\subsection{Version}
A software version in the standard {\tt major.minor.patch} format.

\example\ {\tt 3.2.7}




%
% STANDARDS
%

\section{Standards}\label{standards}

\subsection{ECMA 404}
ECMA 404 describes a lightweight, plain-text format for data
interchange based on a subset of the JavaScript language.

\source{ECMA}{http://www.ecma-international.org/publications/files/ECMA-ST/ECMA-404.pdf}\\
\source{JSON.Org}{http://json.org}


\subsection{ISO 8601}
ISO 8601 is an international standard for describing dates, times,
intervals and repetition.

\source{ISO}{http://www.iso.org/iso/home/store/catalogue_tc/catalogue_detail.htm?csnumber=40874}\\
\source{Wikipedia}{https://en.wikipedia.org/wiki/ISO_8601}

\end{document}
