%
% pScheduler Test Guide for DNS
%

\def\testname{DNS}


% TODO: Add titlepage to the options.
\documentclass[10pt]{article}

\input pscheduler-tex.tex

\DRAFT

\title{pScheduler Test Guide: {\it \testname}}
\author{The perfSONAR Development Team}


\begin{document}
\maketitle


%
% INTRODUCTION
%

\section{Introduction}

The {\tt \testname} test that benchmarks DNS queries.

This document describes schema version {\tt 1}.

\subsection{Participants}

This test includes the following participants:

\begin{center}
\begin{tabular}{|c|c|}
\hline
{\bf Number} & {\bf Role} \\
\hline
{\tt 0} & DNS inquirer \\
\hline
\end{tabular}
\end{center}



%
% TEST SPECIFICATION
%

\section{Test Specification Format}

\subsection{Description}

The test specification consists of a single JSON object containing the
pairs below.  \seejson


\typeditem{nameserver}{Host} Specific nameserver to query.  Defaults to
system default.

\typeditem{record}{String} Whick type of record to request from the
nameserver: a, aaaa, cname, mx, ns, ptr, soa, txt.  Defaults to "a".

\typeditem{schema}{SchemaVersion} Version of the specification format.

\typeditem{source}{Host} The IP to use for the source interface.  Defaults
to the localhost.

\typeditem{query}{Host} The hostname whose record you are querying the
nameserver for.  If this is an IP address, it will be converted to the
in-ddr.arpa format where applicable.

\typeditem{timeout}{Duration} Amount of time to wait for a response
for each packet sent.  Defaults to "PT5S"


\subsection{Example}
\begin{lstlisting}[language=json]
{
    "schema": 1,
    "nameserver": "nameserver.example.com",
    "record": "a",
    "query": "findme.example.com",
    "timeout": "PT10S",
}
\end{lstlisting}


%
% RESULT FORMAT
%

\section{Result Format}

\subsection{Description}
The result consists of a single JSON object containing the pairs
below.  \seejson

\typeditem{schema}{SchemaVersion} Version of the result format.

\typeditem{succeeded}{Boolean} Whether or not the test was a success.

\typeditem{records}{Array} An array of record objects as
described below.

\typeditem{time}{Duration} Length of time resolving the request.

\subsection{record}

Each item in the {\tt records} array of the result consists of a single
JSON object containing the pairs below.  \seejson

\typditem{name}{String} name of record

\typditem{rr_type}{String} type of record ( DNSResultA, DNSResultAAAA,
DNSResultCNAME, DNSResultMX, DNSResultNS, DNSResultPTR, DNSResultSOA,
DNSResultTXT )

\typditem{ttl}{Duration} name of record

\typditem{class}{String} name of record

\subsection{Example}
\begin{lstlisting}[language=json]
{
    "schema": 1,
    "time": "PT1S"
}
\end{lstlisting}



%
% LIMIT FORMAT
%

\section{Limit Format}

\subsection{Description}
Limits for this test consist of a single JSON object containing the
pairs below.  \seelimit

\typeditem{duration}{Limit/Duration} Range of allowed idle times.

\typeditem{starting-comment}{Limit/String} String allowed for the starting comment.

\typeditem{starting-comment}{Limit/String} String allowed for the parting comment.


\example
\begin{lstlisting}[language=json]
{
    "duration": {
        "description": "Short periods",
        "range": {
            "lower": "PT15S",
            "upper": "PT1M"
        },
        "invert": false
    },
    "starting-comment" : {
	"description": "No platypi",
	"match": {
	    "style": "regex",
	    "match": "platypus",
	    "invert": true
	},
	"fail-message": "Starting comment cannot contain 'platypus'",
    },
    "parting-comment" : {
     
	"description": "Require a vowel if not empty''
	"match": {
	    "style": "regex",
	    "match": "(^\$|[aeiou])"
	}
	"fail-message": "Parting comment must contain a vowel if not empty", 
    }
}

\end{lstlisting}


\end{document}
