%
% pScheduler Test Guide for Trace
%

\def\testname{trace}


% TODO: Add titlepage to the options.
\documentclass[10pt]{article}

\input pscheduler-tex.tex

\DRAFT

\title{pScheduler Test Guide: {\it \testname}}
\author{The perfSONAR Development Team}


\begin{document}
\maketitle


%
% INTRODUCTION
%

\section{Introduction}

The {\tt \testname} test determines the route a packet will take from
one host to another.

Note that the name of this test, {\tt \testname}, was selected to
avoid confusion with {\tt traceroute}, the name of a tool which
carries out this test.

This document describes schema version {\tt 1}.

\subsection{Participants}

This test includes the following participants:

\begin{center}
\begin{tabular}{|c|c|}
\hline
{\bf Number} & {\bf Role} \\
\hline
{\tt 0} & Originator \\
\hline
\end{tabular}
\end{center}



%
% TEST SPECIFICATION
%

\section{Test Specification Format}

\subsection{Description}

The test specification consists of a single JSON object containing the
pairs below.  \seejson

\typeditem{schema}{Cardinal} The schema version of this specification.

\typeditem{dest}{Host} The network address or hostname of the receiver.

\stdvalue{ip-version} If {\tt dest} provides clues about the type of
address it represents, that type will be used.  Otherwise, the default
will be {\tt 4}.

\typeditem{probe-type}{String} The type of probe to be used in
determining the path to the destination.  Valid values are {\tt udp}
for UDP (the default), {\tt icmp} for ICMP echos and {\tt tcp} for TCP
SYN.

\typeditem{length}{Cardinal} The length of test packets sent, in
bytes.

\typeditem{fragment}{Boolean} Whether or not fragmentation of the
probe packets should be allowed.  The default is \false.

\typeditem{first-ttl}{Cardinal} What time-to-live value to use
initially.  The default is {\tt 1}.

% TODO:  Do we want to support the -g gateway option?


% TODO: Do this by IP/hostname or interface?  I prefer IP --MF 

\typeditem{source}{Host} The IP to use for the source interface.  The
default behavior is to use the routing table.  Note that the format of
this value must be compatible with the {\tt ip-version} in effect.

\typeditem{hops}{Cardinal} The maximum number of hops that will be
probed.  The default is {\tt 30}.

\typeditem{hostnames}{Boolean} Whether or not attempts should be made
to map IP addresses discovered during the test to hostnames.  The
default is {\tt True}.

\typeditem{dest-port}{IPPort} For the {\tt UDP} and {\tt TCP} probe
types, the destination port to be used.

\typeditem{tos}{IPTOS} Type of service to be used.  \todo{This needs to
  be defined in the dictionary.}

\typeditem{wait}{Duration} The amount of time to wait for a response
to be sent after a probe is sent.

\todo{Support the -r (bypass routing) switch?}

\todo{Support the -s source_addr switch?}

\typeditem{send-wait}{Duration} The amount of time to wait between
probes.

\typeditem{as}{Boolean} Whether or not the AS for each hop should be
determined.  The default is \false.

\todo{How many of the more esoteric {\tt traceroute} options do we
  want to support?}


\subsection{Example}
\todo{Finish this.}
\begin{lstlisting}[language=json]
{
    "schema": 1,
    "dest": "somehost.example.org",
    "ip-version": 4
}
\end{lstlisting}



%
% RESULT FORMAT
%

\section{Result Format}

\subsection{Description}
The result consists of a single JSON object containing the pairs
below.  \seejson

\todo{Write this.}

\subsection{Hop Format}

Each item in the {\tt hops} array of the result consists of a single
JSON object containing the pairs below.  \seejson

Note that  all pairs are  optional, and a  hop for which there  was no
response will be an empty object.

\typeditem{ip}{IPAddress} The IP address of a hop along the path.
\todo{This needs to be defined in the dictionary.}

\typeditem{hostname}{Host} The hostname of the responding host.  This
value will not be provided if {\tt hostnames} in the test
specification is \false.

\typeditem{rtt}{Duration} The round-trip time between the originator
and the responding host.

\typeditem{as}{Cardinal} The autonomous system number of the host if
known and {\tt as} in the test specification is \true.

\typeditem{error}{String} If there was an error reported for this hop,
one of the values enumerated below.

\begin{itemize}
\item ICMP Type 3 errors, from RFC 792:
\subitem {\tt net-unreachable}
\subitem {\tt host-unreachable}
\subitem {\tt protocol-unreachable}
\subitem {\tt port-unreachable}
\subitem {\tt fragmentation-needed-and-df-set}
\subitem {\tt source-route-failed}
\subitem {\tt destination-network-unknown}
\subitem {\tt destination-host-unknown}
\subitem {\tt source-host-isolated}
\subitem {\tt destination-network-administratively-prohibited}
\subitem {\tt destination-host-administratively-prohibited}
\subitem {\tt network-unreachable-for-type-of-service}
\subitem {\tt icmp-destination-host-unreachable-tos}

\item ICMP Type 3 errors, from RFC 1812:
\subitem {\tt communication-administratively-prohibited}
\subitem {\tt host-precedence-violation}
\subitem {\tt precedence-cutoff-in-effect}
\end{itemize}


\subsection{Example}
\begin{lstlisting}[language=json]
{
    "schema": 1,

    "succeeded": true,
    "dest-host": "www.example.org",
    "dest-ip": "203.0.113.80",
    "length": 65536,
    "hops": [
        {
          "ip": "192.0.2.1",
          "hostname": "eng-gw.example.com",
          "rtt": "PT0.000219S",
          "as": 2349
        },
        {
        },
        {
          "ip": "198.51.100.12",
          "hostname": "firewall.example.com",
          "rtt": "PT0.00159S",
          "error": "net-unreachable"
        }
    ]
}
\end{lstlisting}


\end{document}
